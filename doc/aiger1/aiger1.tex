\documentclass[10pt]{llncs}
\usepackage{fancyhdr,color}
\usepackage[margin=1in]{geometry}
\renewcommand{\headrulewidth}{0pt}
\pagestyle{plain}
\cfoot{\color{blue}Technical Report 07/1, October 2007, FMV Reports Series \\
{\small Institute for Formal Models and
Verification, Johannes Kepler University \\
Altenbergerstr.~69, 4040 Linz, Austria}}
\title{The AIGER And-Inverter Graph (AIG) Format Version 20071012}
\author{Armin Biere}
\institute{Johannes Kepler University Linz, Austria}
\parindent0em
\parskip=.5\baselineskip
\begin{document}
\maketitle
\thispagestyle{fancy}

\begin{abstract}
  This report describes the AIG file format as used by the AIGER library.
  The purpose of this report is not only to motivate and document the
  format, but also to allow independent implementations of writers and
  readers by giving precise and unambiguous definitions.
\end{abstract}

\section{Acknowledgements}

  The format went through various incarnations even before real code became
  available.  In particular the following colleagues gave invaluable
  feedback on earlier drafts.
\begin{center}
\begin{minipage}{.87\linewidth}
      Jason Baumgartner, Roderick Bloem, Robert Brummayer, Alessandro
      Cimatti, Koen Claessen, \\ Niklas E\'en, Marc Herbstritt, Geert Janssen,
      Barbara Jobstmann, Hyondeuk Kim, Toni Jussila, \\ Duraid Madina,
      Ken McMillan, Alan Mishchenko, Tobias Nopper, Fabio Somenzi, 
      \\ Niklas S\"orensson, Allen Van Gelder.
\end{minipage}
\end{center}
  We also want to thank Holger Hermanns who started the discussion on having
  a model checking competition affiliated to CAV and also provided strong
  support after the idea became more concrete.

\section{Availability}
  
  You can download up-to date versions of this report, the AIGER library,
  and utilities from here:
  \begin{quote}
  \url{http://fmv.jku.at/aiger}
  \end{quote}

\section{Introduction}

  The name AIGER contains as one part the acronym AIG of And-Inverter
  Graphs and also if pronounced in German sounds like the name of the
  ``Eiger'', a mountain in the Swiss alps.  This choice should emphasize the
  origin of this format. It was first openly discussed at the Alpine
  Verification Meeting 2006 in Ascona as a way to provide a simple, compact
  file format for a model checking competition affiliated to CAV 2007.

  The AIGER format has an ASCII and a binary version.  The ASCII version is
  the format of choice if an AIG is to be saved by an application which
  does not want to use the AIGER library.  It is simple to generate and has
  less restrictions.  The easiest way to write an AIG in AIGER format is of
  course to use the AIGER library.
  
  The binary format is more restricted and much more compact.  It is easier
  to read and thus is the format of choice for benchmarks and competitions.
  The average user can either use the AIGER library to generate a binary
  file directly or if it is impossible to use the library due to LICENSE or
  language restrictions, generate a file in ASCII format and transform
  it into the binary format with the ``\texttt{aigtoaig}'' utility.
  The reverse is also
  possible, if for instance a human readable version of an AIG in binary
  AIGER format is required.  Another usage of the ``\texttt{aigtoaig}'' tool is to
  validate AIGs in binary AIGER format produced by other tools not using
  the AIGER library.

  Let us start with some simple examples in ASCII format which all can be
  found in the ``\texttt{examples}'' subdirectory.  An AIGER file in ASCII format
  starts with the format identifier string ``\texttt{aag}'' for ASCII AIG and 5 non
  negative integers ``\texttt{M}'', ``\texttt{I}'', ``\texttt{L}'', ``\texttt{O}'',
  ``\texttt{A}'' separated by spaces.

  The interpretation of the integers is as follows
\begin{verbatim}
    M = maximum variable index
    I = number of inputs
    L = number of latches
    O = number of outputs
    A = number of AND gates
\end{verbatim}
  The empty circuit without inputs nor outputs consists of the single line
\begin{verbatim}
    aag 0 0 0 0 0               header
\end{verbatim}
  
  The file that consists of the following two lines
\begin{verbatim}
    aag 0 0 0 1 0               header
    0                           output
\end{verbatim}
  represents the constant FALSE.  Note that the comments to the right are
  not part of the file.  The following file represents the constant TRUE:
\begin{verbatim}
    aag 0 0 0 1 0
    1
\end{verbatim}
  The single ``\texttt{1}'' in the header specifies that the number of outputs is one.
  In this case the header is followed by a line which contains the literal
  of the single output.

  The following file is a buffer
\begin{verbatim}
    aag 1 1 0 1 0               header
    2                           input
    2                           output
\end{verbatim}
  and the following an inverter
\begin{verbatim}
    aag 1 1 0 1 0               header
    2                           input
    3                           output          !1
\end{verbatim}
  The maximal variable index is 1, the first number.  The second number
  represents the number of inputs.  The first line consists of the single
  input literal.  A variable is transformed into a literal by multiplying
  it by two.  In the first case the output specified on the last line leaves
  the input unchanged.  In the second case the output inverts the input.  The
  output literal ``\texttt{3}'' has its sign bit, the least significant bit, set to one
  accordingly.
  
  An AND gate looks as follows
\begin{verbatim}
    aag 3 2 0 1 1
    2                           input 0
    4                           input 1
    6                           output 0
    6 2 4                       AND gate 0      1 & 2
\end{verbatim}
  The literal representing the AND gate is ``\texttt{6}'' with variable index
  ``\texttt{3}'',
  the first number in the header, which denotes the maximal variable index.
  The last number is the number of AND gates.

  An OR gate can be formulated as
\begin{verbatim}
    aag 3 2 0 1 1
    2                           input 0
    4                           input 1
    7                           output 0        !(!1 & !2)
    6 3 5                       AND gate 0      !1 & !2
\end{verbatim}
  Let us now turn to a more complete combinational circuit
\begin{verbatim}
    aag 7 2 0 2 3               header line
    2                           input 0         1st addend bit 'x'
    4                           input 1         2nd addend bit 'y'
    6                           output 0        sum bit        's'
    12                          output 1        carry          'c'
    6 13 15                     AND gate 0      x ^ y
    12 2 4                      AND gate 1      x & y
    14 3 5                      AND gate 2      !x & !y
    i0 x                        symbol
    i1 y                        symbol
    o0 s                        symbol
    o1 c                        symbol
    c                           comment header
    half adder                  comment
\end{verbatim}
  The symbol table is optional and does not need to be complete, but may
  only contain symbols for inputs, latches, or outputs.

  Sequential circuits have latches as state elements. Here is a toggle flip
  flop, which has no input, one latch, and two outputs, its current state and
  its negation:
\begin{verbatim}
    aag 1 0 1 2 0
    2 3                         latch 0 with next state literal
    2                           output 0
    3                           output 1
\end{verbatim}
  Latches are always assumed to be initialized to zero.  The same toggle
  flip flop with an enable and additional explicit active low reset input:
\begin{verbatim}
    aag 7 2 1 2 4
    2                           input 0         'enable'
    4                           input 1         'reset'
    6 8                         latch 0         Q next(Q)
    6                           output 0        Q
    7                           output 1        !Q
    8 4 10                      AND gate 0      reset & (enable ^ Q)
    10 13 15                    AND gate 1      enable ^ Q
    12 2 6                      AND gate 2      enable & Q
    14 3 7                      AND gate 3      !enable & !Q
\end{verbatim}
  The order of the literals and the definitions of the AND gates
  is irrelevant.  The binary format described more formally below places
  more restrictions on the order and also does not allow unused literals.

\section{Design Choices}
  
  The format should allow to model combinational circuits.

  Structural SAT problems can be described.
  
  The format should allow to model sequential circuits.

  Model checking problems can be described.

  The operators are restricted to bit level.

  The set of operators needs to be as simple as possible.

  A compact standardized binary format should be available.

  The ASCII format should be as easy as possible to write by programs.

  The binary format should be as easy as possible to read by programs.

  A symbol table and comments can be included.

  Symbol table and comments can be ignored reading the file sequentially.

  Some simple form of extensibility should be possible.

\section{Header}

  The AIGER format describes circuits by multi-rooted And-Inverter Graphs
  (AIGs).  A file in AIGER format has to start with a format identifier
  string, which is either ``\texttt{aag}'' for ASCII format or ``\texttt{aig}'' for the binary
  format.  After the format identifier string and one space character
  follow 5 non negative integers ``\texttt{M}'', ``\texttt{I}'', ``\texttt{L}'',
  ``\texttt{O}'', and ``\texttt{A}'' in ASCII
  encoding.  The maximal variable index ``\texttt{M}'' is the first number in the
  header.  The circuit has ``\texttt{I}'' inputs, ``\texttt{L}'' latches,
  ``\texttt{O}'' outputs, and
  consists of ``\texttt{A}'' AND gates.  If all variables are used and there are no
  unused AND gates then ``\texttt{M = I + L + A}''.

  An unsigned respectively non negative integer is either ``\texttt{0}'' or a strictly
  positive ASCII encoded digit followed by a sequence of arbitrary digits
  including ``\texttt{0}''.  The ``\texttt{a}'' of the format identifier string is the first
  character of the file.  The format identifier string ``\texttt{aag}'' respectively
  ``\texttt{aig}'' and the numbers are all separated by exactly one space character.
  The header line ends with a new line character immediately after the
  last digit of the number of ands ``\texttt{A}''.

\section{Variables and Literals}

  Literals are constants or signed variables and are also represented by
  unsigned integers.  The least significant bit of the unsigned word
  encoding a literal is the sign bit.  The remaining bits represent the
  variable index.
  
  In essence, to obtain the sign bit of a literal we take its unsigned
  integer representation modulo 2.  A sign bit of one means negated, a sign
  bit of zero unnegated.  To extract the variable from a literal we divide
  it by 2.  To obtain a literal from a variable in the other direction we
  multiple the variable index by 2 and optionally add 1 if the variable
  should be negated.   A literal can be negated by just toggling its least
  significant bit.
  
  The constant FALSE is represented by the literal ``\texttt{0}'', the constant TRUE by
  ``\texttt{1}''.  This implies, that FALSE is unnegated, while TRUE is negated.

  AIGER only models cycle-accurate models.  Latches are assumed to have a
  reset zero state, e.g. are initialized to zero and no explicit reset nor
  clock signal is necessary.  More general circuit models with non zero
  reset states or even non deterministic transition relations can be
  translated into the AIGER format by simple constructions.  See for
  instance the techniques implemented in ``\texttt{smvtoaig}'' to translate 
  arbitrary flat and boolean encoded SMV models into AIGER format.

\section{ASCII Format}
  
  The ASCII format has ``\texttt{aag}'' as format identifier in the header, which is an
  acronym for ASCII AIG.  The names of files in the ASCII format are
  supposed to have an ``\texttt{.aag}''  extension.  The AIGER library also supports
  files compressed with GNU GZIP.  In this case an additional ``\texttt{.gz}'' suffix
  is expected.

  After the header the ``\texttt{I}'' inputs are listed, one per line, as unnegated
  literal, e.g.  represented as even positive integers.  Then the ``\texttt{L}''
  latches are defined.  Again one latch per line.  Each line consists of
  exactly two positive integers separated by a space character.  The first
  is even and denotes the current state of the latch. The second is the
  literal that defines the next state function of this latch.  
  
  Then the ``\texttt{O}'' output literals are listed, one per line.  Here arbitrary
  literals are allowed.  This concludes the interface part and the
  definitions of the ``\texttt{A}'' AND gates follow.

  The definition of an AND gate consists of three positive integers all
  written on one line and separated by exactly one space character.  The
  first integer is even and represents the literal or left-hand side (LHS).
  The two other integers represent the literals of the right-hand side
  (RHS) of the AND gate.

  In order to be well formed, the ``\texttt{I}'' inputs, the current state literal of
  the ``\texttt{L}'' latches, e.g. the pseudo-primary inputs, and the LHS of the
  ``\texttt{A}''
  AND gates are all different and define exactly ``\texttt{I + L + A}'' literals.  The
  other literals, except for the two constants ``\texttt{0}'' and ``\texttt{1}'', are undefined
  and can not be used as output, as next state literal, nor on the RHS of an
  AND.  Furthermore, the definitions of the ANDs have to be acyclic.  To be
  more precise: The literal on the LHS of an AND is defined to depend on the
  literals on the RHS after stripping sign bits.  The transitive non
  reflexive closure of this dependency relation has to be acyclic.

  In the ASCII format both checking for undefined literals and checking for
  cyclic dependencies has to be done explicitly.  This is the price one has
  to pay for a simple, easy to generate and less restricted format.  The
  binary format on the other hand has a strict order requirement on the
  literals not only for inputs and latches, but also for the RHS literals
  with respect to the LHS literals.  The binary format does not require such
  an explicit check. The AIGER library and all the tools including
  ``\texttt{aignm}''
  and ``\texttt{aigtoaig}'', which read AIGs in AIGER format with the AIGER library,
  can be used to validate AIGER files in ASCII or binary format.

  Finally note, that in the ASCII format the number of defined literals does
  not have to match the maximal variable index ``\texttt{M}''.  In essence, some
  literals may just not be used in ASCII format.

\section{Symbols}
  
  After the definitions of the AND gates an optional symbol table may
  follow.  A symbol is an arbitrary ASCII string of printable characters
  excluding the new line character.  Symbols can only be attached to inputs,
  latches and outputs and there is at most one symbol per input, latch or
  output allowed.  A symbol entry makes up one line in the input and
  consists of a symbol type specifier which is either ``\texttt{i}'',
  ``\texttt{l}'', or ``\texttt{o}'' on
  the first character position, followed by a position, which is not
  separated by a space.  After a space character the symbol name starts and
  continues until but not including the next new line character.  Therefore
  a symbol table entry looks as follows:
\begin{verbatim}
    [ilo]<pos> <string>
\end{verbatim}
  The position ``\texttt{pos}'' of the symbol denotes the position of the input, latch,
  or output, in the list of inputs, latches, and outputs respectively.  It
  has to follow immediately the symbol type identifier without space.

  The symbol table is put after all the definitions, such that an
  application reading an AIG can just stop reading the file after it has
  read the definitions.  The same applies to comments which come last.

\section{Comments}
 
  After the symbol table an optional comment section may start.  The comment
  section starts with a ``\texttt{c}'' character followed by a new line.  The following
  lines are comments.  Each comment starts at the first character of a line
  and extends until the next new line character.  There can be no comment
  lines at all, but the last comment has to be terminated by a new line
  character, which then also has to be the last character of the file.

\section{Binary Format Motivation}

  The binary format is more restricted than the ASCII format.  Literals have
  to follow a specific order.  The order restrictions and using a
  two-complement binary representation of numbers makes the binary format
  not only easier to read but also much more compact.  Experiments show that
  these restrictions and an additional delta encoding result in smaller
  uncompressed files.  The binary format is often smaller than the same
  model in GZIP compressed ASCII format.  Without symbol tables the
  reduction is typically a factor of two compared to the GZIP compressed
  ASCII format.  Compressing the binary format results in additional
  reduction.  
  
  As example consider the following SMV model ``\texttt{texas.parsesys\^\relax 1.E.smv}'', the
  largest benchmark from the TIP suite:  
\begin{verbatim}
    1204637 unmangled.smv
     485595 mangled.smv
     191045 unstripped.aag
     185098 stripped.aag
     114796 unmangled.smv.gz
      97455 mangled.smv.gz
      72486 unstripped.aag.gz
      70693 stripped.aag.gz
      44044 unstripped.aig
      38097 stripped.aig
      28061 unstripped.aig.gz
      26226 stripped.aig.gz
\end{verbatim}
  After flattening and binary encoding, the SMV model has a size of 1204637
  bytes (unmangled.smv).  Simply renaming the symbols, e.g. mangling them,
  results in a reduction in file size of almost a factor of 3.  At least
  another factor of two can be obtained by translating the original
  unmangled SMV file into an AIG and writing it in the ASCII format
  (unstripped.aag) with our tool ``\texttt{smvtoaig}''.  This file still contains all
  the unmangled symbols.  Stripping the symbol table reduces the size
  slightly (stripped.aag).  Compressing these four files with GZIP results
  in another reduction of at least 2.  In particular note, that the
  uncompressed and unstripped binary format is two third the size of the
  compressed and stripped ASCII format.  Finally using the binary format,
  gives almost an additional factor of two.  Compressing the binary format
  gives further reduction.

  For the whole TIP suite the sizes are as follows
\begin{verbatim}
    4393611 unmangled.tar.gz
    3099882 mangled.tar.gz
    2676172 aigtosmv.tar.gz
    2303631 aag.tar.gz
     347325 aig.tar.gz
\end{verbatim}
  Here we stripped the symbols in the AIG case.  The tar file
  ``\texttt{aigtosmv.tar.gz}'' contains the result of translating the files from binary
  format (tip/aig.tar.gz) back into SMV format.  It is also remarkable, that
  with the exception of the binary format, GZIP can not make use of the fact
  that most of the models in the TIP suite occur multiple times though with
  different properties.

  Finally, as a combinational example consider the CNF benchmark
  ``\texttt{f10bidw}''
  submitted by P. Manolios and S. K. Srinivasanused to the SAT Race 2006
  with 800k variables and roughly 2.4 million clauses.  Since the original
  benchmark consists of only AND gates and one unit clause, we could easily
  extract a corresponding AIG with our tool ``\texttt{cnf2aig}''.
\begin{verbatim}
    55125988 f10bidw.cnf
    17728435 f10bidw.aag
    11743646 f10bidw.cnf.gz
     5827259 f10bidw.aag.gz
     2904681 f10bidw.aig
     1931015 f10bidw.aig.gz
\end{verbatim}
  These experiments show that using a binary format can considerably reduce
  space requirements for storing benchmarks.  This reduction is even more
  important if the AIGER format is used for large combinational benchmarks,
  e.g. in a SAT context as in the last example.
  
\section{Binary Format Definition}

  The binary format is semantically a subset of the ASCII format with a
  slightly different syntax.  The binary format may need to reencode
  literals, but translating a file in binary format into ASCII format and
  then back in to binary format will result in the same file.

  The main differences of the binary format to the ASCII format are as
  follows.  After the header the list of input literals and all the
  current state literals of a latch can be omitted.  Furthermore the
  definitions of the AND gates are binary encoded.  However, the symbol
  table and the comment section are as in the ASCII format.

  The header of an AIGER file in binary format has ``\texttt{aig}'' as format
  identifier, but otherwise is identical to the ASCII header.  The standard
  file extension for the binary format is therefore ``\texttt{.aig}''. 
  
  A header for the binary format is still in ASCII encoding:
\begin{verbatim}
    aig M I L O A
\end{verbatim}
  Constants, variables and literals are handled in the same way as in the
  ASCII format.  The first simplifying restriction is on the variable
  indices of inputs and latches.  The variable indices of inputs come first,
  followed by the pseudo-primary inputs of the latches and then the variable
  indices of all LHS of AND gates:
\begin{verbatim}
    input variable indices        1,          2,  ... ,  I
    latch variable indices      I+1,        I+2,  ... ,  (I+L)
    AND variable indices      I+L+1,      I+L+2,  ... ,  (I+L+A) == M
\end{verbatim}
  The corresponding unsigned literals are
\begin{verbatim}
    input literals                2,          4,  ... ,  2*I
    latch literals            2*I+2,      2*I+4,  ... ,  2*(I+L)
    AND literals          2*(I+L)+2,  2*(I+L)+4,  ... ,  2*(I+L+A) == 2*M
\end{verbatim}
  All literals have to be defined, and therefore ``\texttt{M = I + L + A}''.  With this
  restriction it becomes possible that the inputs and the current state
  literals of the latches do not have to be listed explicitly.  Therefore,
  after the header only the list of ``\texttt{L}'' next state literals follows, one per
  latch on a single line, and then the ``\texttt{O}'' outputs, again one per line.

  In the binary format we assume that the AND gates are ordered and respect
  the child parent relation.  AND gates with smaller literals on the LHS
  come first.  Therefore we can assume that the literals on the right-hand
  side of a definition of an AND gate are smaller than the LHS literal.
  Furthermore we can sort the literals on the RHS, such that the larger
  literal comes first.  A definition thus consists of three literals
\begin{verbatim}
      lhs rhs0 rhs1
\end{verbatim}
  with ``\texttt{lhs}'' even and ``\texttt{lhs > rhs0 >= rhs1}''.  Also the variable indices are
  pairwise different to avoid combinational self loops.  Since the LHS
  indices of the definitions are all consecutive (as even integers),
  the binary format does not have to keep ``\texttt{lhs}''.  In addition, we can use
  the order restriction and only write the differences ``\texttt{delta0}'' and
  ``\texttt{delta1}''
  instead of ``\texttt{rhs0}'' and ``\texttt{rhs1}'', with
\begin{verbatim}
      delta0 = lhs - rhs0,  delta1 = rhs0 - rhs1
\end{verbatim}
  The differences will not be negative, and in practice often very small.
  We can take advantage of this fact by the simple little-endian encoding of
  unsigned integers of the next section.  After the binary delta encoding of
  the RHSs of all AND gates, the optional symbol table and optional comment
  section start in the same format as in the ASCII case.

\section{Binary Encoding of Deltas}
  
  Assume that ``\texttt{w0, ..., wi}'' are 7-bit words, ``\texttt{w1}'' to
  ``\texttt{wi}'' all non zero and
  the unsigned number ``\texttt{x}'' can be represented as
\begin{verbatim}
    x = w0 + 2^7*w1 + 2^14*w2 + 2^(7*i)*wi
\end{verbatim}
  The binary encoding of x in AIGER is the sequence of i bytes b0, ... bi:
\begin{verbatim}
    1w0, 1w1, 1w2, ..., 0wi
\end{verbatim}
  The MSB of a byte in this sequence signals whether this byte is the last
  byte in the sequence, or whether there are still more bytes to follow.
  Here are some examples:
\begin{verbatim}
      unsigned integer   byte sequence of encoding (in hexadecimal)
 
                     x   b0 b1 b2 b3
                        
                     0   00
                     1   01
      2^7-1    =   127   7f
      2^7      =   128   80 01
      2^8  + 2 =   258   82 02
      2^14 - 1 = 16383   ff 7f
      2^14 + 3 = 16387   83 80 01
      2^28 - 1           ff ff ff 7f
      2^28 + 7           87 80 80 80 01
\end{verbatim}
  This encoding can reduce the number of bytes by at most a factor of 4,
  which very often in practice is almost reached, in particular in our
  application where many small numbers are expected.
  
  This binary encoding of arbitrary precision unsigned integers is
  platform-independent and thus 64-bit clean.  Unfortunately, this is not
  true for the following code snippets in C.  We also just ignore overflows
  and file errors, but otherwise this code shows that encoding and decoding
  is very simple.
\begin{verbatim}
    unsigned char
    getnoneofch (FILE * file)
    {
      int ch = getc (file);
      if (ch != EOF)
        return ch;

      fprintf (stderr, "*** decode: unexpected EOF\n");
      exit (1);
    }

    unsigned
    decode (FILE * file)
    {
      unsigned x = 0, i = 0;
      unsigned char ch;

      while ((ch = getnoneofch (file)) & 0x80)
        x |= (ch & 0x7f) << (7 * i++);

      return x | (ch << (7 * i));
    }

    void
    encode (FILE * file, unsigned x)
    {
      unsigned char ch;

      while (x & ~0x7f)
        {
          ch = (x & 0x7f) | 0x80;
          putc (ch, file);
          x >>= 7;
        }
     
      ch = x;
      putc (ch, file);
    }
\end{verbatim}
  Not checking that the next character is an EOF may result in an infinite
  loop.  In the binary format of AIGER we always expect a complete sequence.
  Therefore if an EOF is read before the sequence is complete a parse error
  occurs.  A simple solution to this problem is to check the return value of
  ``\texttt{getc}''.  If the value is EOF then abort decoding with a parse error.

\section{Property Checking}
  
  The AIGER format can easily be used for various types of property
  checking.  A combinational circuit with no latches, e.g. L = 0, and
  exactly one output can be assumed to be a circuit for which we want to
  force the output to be one.  This is an encoding of SAT.

  For sequential circuits model checking of simple safety properties is
  encoded in the same way.  To check liveness we interpret each of the
  outputs as fairness constraint.  An algorithm that finds a reachable fair
  cycle for these fairness constraints allows to model check LTL properties.

  In addition we plan to support PSL properties in the following way.
  PSL properties are written in a separate file.  As atomic properties
  only output of the circuit are allowed.  The outputs are referenced
  through a mandatory symbolic name in the symbol table.

\section{Vectors, Stimulus, Traces, Solutions and Witnesses}

  In this section we define the semantics and syntax of traces and solutions
  to property checking problems.  More specifically we only consider
  structural SAT solving and bad state detection witnesses.
  
  In essence, a valid solution is a list of input vectors. An input vector
  may contain beside ``\texttt{0}'' and ``\texttt{1}'' also ``\texttt{x}'' values.  Such a list of input
  vectors is a valid witness iff for any instantiation of the ``\texttt{x}'' values by
  ``\texttt{0}'' or ``\texttt{1}'' two-valued simulation will produce at least one
  ``\texttt{1}'' at the
  output of the AIG, assuming that the AIG starts in its initial all ``\texttt{0}''
  state.  For structural SAT problems, the AIG does not contain latches.
  In this case one input vector is enough.

  In principle one can use three valued logic for fast simulation of AIGs
  under partial two-valued input assignments.  Beside two-valued logic
  constants ``\texttt{0}'' and ``\texttt{1}'', we also use the logic constant
  ``\texttt{x}''.  Operations in
  this three-valued logic are as usual:
\begin{verbatim}
    a !a     & 0 1 x
    0  1     0 0 0 0
    1  0     1 0 1 x
    x  x     x 0 x x
\end{verbatim}
  A vector of values or short vector of size ``\texttt{n}'' is a list of
  ``\texttt{n}''
  three-valued constants.  A vector of size ``\texttt{n}'' is represented by
  an ASCII string of length ``\texttt{n}'' containing only the characters
  ``\texttt{0}'',
  ``\texttt{1}'', and ``\texttt{x}''.

  In the following fix an AIG in AIGER format of type ``\texttt{M I L O A}''.  An input
  vector is a vector of size ``\texttt{I}'', an output vector is a vector of size
  ``\texttt{O}'',
  and a state vector is a vector of size ``\texttt{L}''.

  A stimulus for an AIG in AIGER format is a list of input vectors.
  The file format for a stimulus consists of lines of ASCII strings that
  represent the input vectors separated and terminated by exactly one new
  line character.

  A transition contains, in this order, a current state, input, output, and
  next state vector.  A transition is consistent with the logic of the given
  AIG if the result of simulating the current state vector under the input
  vector in three-valued logic produces the given output and next state.  In
  the ASCII representation of a transition, the strings representing the
  four vectors are separated by space characters, exactly one space
  character between two strings.  This implies that a transition for a
  combinational circuit with ``\texttt{L = 0}'' starts and ends with one space
  character.  A combinational circuit without any input would even have two
  leading space characters.

  A trace consists of a list of transitions.  A trace is consistent if all
  its transitions are consistent, and the current state vector of all
  transitions except the first has the same value as the next state vector
  of its previous transition.  A trace is initialized if the current state
  vector of the first transition only contains ``\texttt{0}''.  Each line of a trace
  file consist of exactly one transition.  Again we use exactly one new line
  character to separate and terminate the list of lines.
  
  Note that we only consider ``cycle accurate simulation'' here.  Furthermore
  the definitions also make sense for combinational circuits with ``\texttt{L =
  0}''.
  In this case, the state vectors will just be vectors of length 0 and
  represented by empty strings.

  A simulator thus takes as input an AIG and a stimulus matching the type
  of the AIG and produces an initialized and consistent trace.  A randomized
  simulator will use randomized input vectors and does not need a stimulus.

  The ``\texttt{x}'' value should not be interpreted as don't care.  For instance, if
  the literal ``\texttt{l}'' is assigned to the value ``\texttt{x}'', then
  ``\texttt{l \& !l}'' produces ``\texttt{x}''
  in three-valued simulation and not ``\texttt{0}''.  This standard semantic of
  three-valued logic allows linear time simulation of a circuit under a
  total three-valued assignment to the inputs.  Simulating partial
  assignments in two-valued logic, more specifically checking whether an
  output is forced to a certain value, is an NP complete problem.  On the
  other hand three-value simulation is purely syntactic.  For instance,
  optimizing an AIG does not need to preserve three-valued simulation
  semantics.

  Because of this syntactic nature of three-valued simulation we define the
  concept of a grounded stimulus, which does not contain ``\texttt{x}'' values.
  A stimulus is an instance of another stimulus if the former is identical
  to the latter, except where the latter has ``\texttt{x}'' values.  Instead of
  ``\texttt{x}'' the
  former can have any of the three values (``\texttt{0}'', ``\texttt{1}'', or ``\texttt{x}'').

  A witness that a bad state can be reached and also a solution to a
  structural SAT problem is in both cases just a not necessarily grounded
  stimulus for which any grounded instance produces through simulation a
  trace in which the output vector of at least one transition contains
  ``\texttt{1}''.
  This simulation is of course over two-valued logic.  The whole discussion
  on three-valued logic and simulation is only included here to explain that
  even though that 'x' can be used in a stimulus, we do not actually use
  three-valued semantics in the definition of a witness.  In the context
  of ATPG, or where reset states are unknown (which would differ from AIG
  semantics), these definitions using three-valued semantics are still
  useful.

  In order to distinguish successful from failing model checking
  respectively SAT solver runs we define the concept of solutions.  A
  solution file may contain a result line and an optional stimulus part.
  The result line either is made up of the single character '0' to denote
  that the solver proved that the output of the AIG can never become one or
  '1' which means that the solver produced a witness that at least one
  output can become one starting from the initial state.  Invalid result
  lines or empty files are interpreted as unknown.  If the result is '1'
  then the rest of the file should contain a witness as defined in the
  previous paragraph.

  The definition of a witness for the existence of a fair cycle is not
  covered yet.

\section{Proposal for Future Extensions}

  There was not much enthusiasm about an earlier proposal to extend AIGER to
  the word level.  Instead, we plan to include two other things for the next
  major version of AIGER.  First there is a suggestion by Alan Mishchenko to
  encode with binary delta encoding not only the AND gates but also the next
  state and output section.  The reason is that in large industrial
  applications, the size of that part can be large.
  
  Furthermore, we plan to add in a downward compatible way, optional
  sections.  The number of literals of these sections would be listed after
  the MILOA numbers.  These sections contain references to as many literals
  as the number denotes and have the semantics of 'probes' into the circuit.
  These probes, we also call them 'secondary outpus', are not part of the
  input/output behavior, but for instance a simplifier has to treat them as
  ordinary outputs.  A synthesis tool can just ignore these sections.
  
  We will define a couple of standard sections, for constraints resp.
  assumptions, fairness constraints, and maybe even for representing
  structural QBF problems.  The intention is that users of the AIGER format
  can use their own section types or even intrepret 'standard' sections
  differentely.
  
\section{Related Work}

  AIGs and similar data structures have been around for a while.  AIGs are
  described in [4,5], RBCs in [1] and BEDs in [2].  Recently AIG based
  algorithms have been used very successfully as alternative to classical
  synthesis algorithms in [6] using ideas from [3].  The former paper
  also contains additional references not listed here.

\begin{description}

\item[1] P. Abdulla, P. Bjesse, and N. Een.
  Symbolic reachability analysis based on SAT-solvers.
  In Proc.~TACAS'00.

\item[2] H. Andersen and H. Hulgaard. 
  Boolean expression diagrams.
  In Proc.~LICS'97.

\item[3] P. Bjesse and A. Boralv.
  DAG-Aware Circuit Compression For Formal Verification.
  In Proc.~ICCAD'04.

\item[4] A. Kuehlmann, V. Paruthi, F. Krohm, and M. Ganai.
  Robust boolean reasoning for equivalence checking and functional 
  property verification.
  IEEE Trans.~on CAD, 21(12), 2002. 

\item[5] A. Kuehlmann, M. Ganai, and V. Paruthi. 
  Circuit-based Boolean Reasoning. 
  In Proc.~DAC'01.

\item[6] A. Mishchenko, S. Chatterjee, and R. Brayton. 
  DAG-Aware AIG Rewriting - A Fresh Look at Combinational Logic Synthesis. 
  In Proc.~DAC'06. 

\end{description}

\end{document}
