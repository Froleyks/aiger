\documentclass{llncs}
\usepackage{color}
\title{
{\color{blue}\large DRAFT}\\
  AIGER 1.9 and Beyond \\
{\color{blue}\large May 30, 2011}}
\author{Armin Biere\inst{1} \and
Keijo Heljanko\inst{2} \and
Siert Wieringa\inst{2}}
\institute{Johannes Kepler University, Austria \and Alto University, Finland}
\begin{document}
\maketitle
\begin{abstract}
This is a short note on the differences between AIGER format version 20071012
and the new versions starting with version 1.9.
\end{abstract}
To ease the transition, the new 1.9 series of AIGER is intended to be
syntactically upward compatible with the previous format, but contains
already all the new features of the upcoming AIGER version 2.0 format.  The
future AIGER 2.0 which will not be syntactically upward compatible, because
it uses a new binary encoding, but then, initially at least, will not have
new features.

For the HWMCC11 competition we will accept tools that work with any of
the three formats.  However, for the new tracks with multiple properties and 
particularly liveness only the new formats (1.9 and 2.0) are supported.

For the upcoming version 2.0 there will be a new format report. Until it is
available, the format report for version 20071012 and this note serve as
language definition for pre 2.0 AIGER.  

In essence there are five new semantic features:

\begin{itemize}
\item reset logic
\item multiple properties
\item invariant constraints
\item liveness properties
\item fairness constraints
\end{itemize}

We will discuss all of them in seperate sections, including syntactic
extensions to the old format.   Then the API changes are considered followed
by the new witness format.

\section{Reset Logic}

As AIGER is also used as intermediate language in synthesis, and
uninitialized latches occur frequently and should be marked as such, we
added support for reset logic.  In the new format, a latch is either
initialized to 0 (as in the old format, now the default), initialized to 1,
or it is unitialized.

Syntactically the line in the AIGER format that defines the next state
literal of a latch might now optionally contain an additional literal, which
is either '0', '1', or the literal of the latch itself.  The former are used
for constant initialization and the latter to define an uninitialized latch.

This allows for future more general reset logic without syntactic changes.
But in the current benchmarks we only support either constant initialized or
uninitialized latches.

\section{Multiple Properties}

A common request from industry was to allow using and checking multiple
properties for the same model.  In practice a model rarely only has one
property and in addition properties like invariants that hold on the model
might help to prove other properties faster.

The extension is rather straigth forward.  Where one property was listed,
i.e. one output in the old format, now multiple properties can be listed.
The major change is in the witness format.  See below for more information
about that.

The other minor change in the first line of the header.  As in the previous
format it has the form ''\texttt{aig} $M$ $I$ $L$ $O$ $A$'', but can not be
extended with four more numbers ``$B$ $C$ $J$ $F$'', where $B$ denotes the
number of ``bad'' properties, and $C$, $J$, $F$ the number of
invariant constraints, justice properties, respectively fairness
constraints.  The latter three are explained below.

The idea is that the new header is an extension of the old header,
but for the extension it is possible to drop a suffix that only contains
zeros.  As example consider the following 1-bit counter with one enable
input (literal 2),  the bad property is the latch output (literal 4), which
is initialized to 0 and has 10 as next state literal.  It flips the latch if
the input is 1.  Otherwise it does not change.  This is realized by the XOR
implemented with 3 AND gates.
{\small
\begin{verbatim}
aag 5 1 1 0 3 1
2
4 10 0
4
6 5 3
8 4 2
10 9 7
\end{verbatim}}
For the examples we use the ASCII
version wither header \texttt{aag}.  The difference to the binary format
with header \texttt{aig} is as before.  In the binary format
the input section (literal 2) and the first literal of the latch definitions
(literal 4) can be dropped.  The AND gates are binary encoded.

In the old format there were no bad sections, the last 1 above.  Bad
properties were listed as outputs.  Thus in the old format the example would
only have a different first list:
{\small
\begin{verbatim}
aag 5 1 1 1 3
\end{verbatim}}

\section{Invariant Constraints}

For the example above we might want to check, wether it is possible to reach
a bad state, in which the latch is 1, without ever enabling the input.
This invariant constraint is supposed to hold from the first state
until and including where the bad state is found.  In linear temporal logic
(LTL) a witness for such a bad state is 
`` $ \mathit{c} \mathrel{\mathbf{U}} (c \wedge b) $ '',
where $c$ is the conjunction of the invariant constraints, and $b$ is
one bad property.  

Witnesses for bad properties are essentially finite paths
and can thus be found with safety property checking algorithms without
further translations.

Positively, negating the bad property, we actually try to prove that the LTL
formula
`` $ \mathit{\bar c} \mathrel{\mathbf{U}} (c \to g) $ '' holds on
all initialized paths, where $g = \bar b$ (good) is the negation of the
considered bad property.  Read ``when $c$ stops to hold it releases
(at this very moment) $g$ to hold if $c$ holds''.

The stronger $(\mathrel{\mathbf{G}} c) \wedge
 \mathrel{\mathbf{G}}g$ requires to check that after a bad state has been
reached, without violating the invariant constraints, it still is possible
to extend the path to an infinite path on which $c$ holds all the time.
This might be really complicated, if for instance the constraints restricts
latch values.  Then extending the finite path to an infinite path requires
to solve another PSPACE hard problem.

Even though this stronger version is the standard semantics of combining
\texttt{INVAR} sections with safety properties in SMV, we prefer the weaker
version, which is easier to check and has simple (finite path) semantics.

Note that an infinite path might still be considered as a witness of a bad
property.  Actually only a finite prefix until and including the bad state
is sufficient.

\section{Liveness Properties and Fairness Constraints}

We assume that the reader is familar with translations of LTL into
generalized B\"{u}chi automata, similar to the LTL2SMV tool.
The result of such a translation is a set of fairness constraints for each
LTL property.  We call such a set a ``justice'' property.  A witness for a
justice property is an infinite initialized path, on which each
fairness constraint in the set is satisfied infinitely often.

Such a set in AIGER is represented as a list of literals.  In particular, we
optionally allow one global justice property, respectively a global list of
$F$ fairness constraints, as last section.  The justice section contains $J$
numbers, where the $i$'th number denotes the number of (local) fairness
constraints of the $i$'th justice property.  After the sizes of all the
justice properties, the literals of the first justice property are listed,
followed by the literals of the second justice property and so on.

As discussed above, fairness constraints do not apply to bad properties, even
though both are listed for the same model.  In particular, if there are only
bad properties, but no justice properties, fairness constraints are actually
redundant.

\section{API}

The library API in 'aiger.h' is extended with functions to support new
features but the already existing functions do not change their meaning.

For the new features the following functions have been added:
{\small
\begin{verbatim}
void aiger_add_reset (aiger *, unsigned lit, unsigned reset);
void aiger_add_bad (aiger *, unsigned lit, const char *);
void aiger_add_constraint (aiger *, unsigned lit, const char *);
void aiger_add_justice (aiger *, unsigned size, unsigned *, const char *);
void aiger_add_fairness (aiger *, unsigned lit, const char *);
\end{verbatim}}

There is an additional '\texttt{reset}' field for latches, as well as 
seperate '\texttt{bad}', '\texttt{constraints}', '\texttt{justice}' and
'\texttt{fairness}' sections.  Each section
is a list of '\texttt{aiger\_symbol}' entries:

{\small
\begin{verbatim}
  struct aiger_symbol
  {
    unsigned lit;                 /* unused for justice */
    unsigned next, reset;         /* used only for latches */
    unsigned size, * lits;        /* used only for justice */
    char *name;
  };
\end{verbatim}}

\section{Witness Format}

The witness format has been adapted to the new features as follows.
First the output file might contain an arbitrary number of witnesses.

A witness starts with either 0, 1, or 2, where 0 means that the property can
not be satisfied, i.e. there is no reachable bad state respectively an
infinite path satisfying the justice property and all the invariant
constraints.  A 2 denotes unknown status, while 1 means that a witness
has been found.

The second line contains the properties satisfied by the witness.
A bad state property is referred to with ``\texttt{b}i'' and a
justice property by ``\texttt{j}i'', where $i$ ranges over the bad
respectively justice property indices, which start at 0.\footnote{This is 
the same convention in the symbol table}.

There might be unitialized latches.  Therefore it is required that the third
line contains the initial state, represented as a list of '0' and '1' ASCII
characters.  The following lines contain input vectors as in the old format.

As all properties and constraints might refer to inputs, i.e.~they are ``Mealy''
outputs, the number of input vectors is the same as the number of states.
Thus a witness contains at least one input vector line.

To seperate witnesses and catch early termination of model checkers, in the
process of printing a full witness, we require that each witness is
terminated with '.' character on a seperate line.

Thus the format looks like follows
{\small
\begin{verbatim}
  { '0' | '1' | '2' }       <newline>          status
( { 'b' | 'c' } <index> ) + <newline>          properties
  { '0' | '1' | 'x' } *     <newline>          initial state
( { '0' | '1' | 'x' } *     <newline> ) +      input vector(s)
\end{verbatim}}

The 'x' characters denote a don't care value.  For the competition
we require that grounding to an arbitrary value (independently) will
always produce a valid witness, but only check that grounding them
to 0 does it. 

Comments start with 'c' and extend until the end of the line.
After filtering them out they are interpreted as new line seperators.

For justice properties the state reached after the last input vector
has to occur before.  It is not mandatory to specify the
loop start.  It is calculated by the simulator.

\section{Acknowledgements}

Acknowledgements go to all the supporters of AIGER and the HWMCC.
The new format report version 2.0 will contain a complete list.

\end{document}
