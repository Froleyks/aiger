\documentclass{llncs}
\usepackage{color}
\title{
{\color{blue}\large DRAFT}\\
  AIGER 1.9 and Beyond \\
{\color{blue}\large May 30, 2011}}
\author{Armin Biere\inst{1} \and
Keijo Heljanko\inst{2} \and
Siert Wieringa\inst{2}}
\institute{Johannes Kepler University, Austria \and Alto University, Finland}
\begin{document}
\maketitle
\begin{abstract}
This is a short note on the differences between AIGER format version 20071012
and the new versions starting with version 1.9.
\end{abstract}
To ease the transition, the new 1.9 series of AIGER is intended to be
syntactically upward compatible with the previous format, but contains
already all the new features of the upcoming AIGER version 2.0 format.  The
future AIGER 2.0 which will not be syntactically upward compatible, because
it uses a new binary encoding, but then, initially at least, will not have
new features.

For the HWMCC11 competition we will accept tools that work with any of
the three formats.  However, for the new tracks with multiple properties and 
particularly liveness only the new formats (1.9 and 2.0) are supported.

For the upcoming version 2.0 there will be a new format report. Until it is
available, the format report for version 20071012 and this note serve as
language definition for pre 2.0 AIGER.  

In essence there are four new semantic features:

\begin{itemize}
\item reset logic
\item multiple properties
\item environment constraints
\item liveness properties
\end{itemize}

We will discuss all of them in seperate sections, including syntactic
extensions to the old format.   Then the API changes are considered followed
by the new witness format.

\section{Reset Logic}

As AIGER is also used as intermediate language in synthesis, and
uninitialized latches occur frequently and should be marked as such, we
added support for reset logic.  In the new format, a latch is either
initialized to 0 (as in the old format, now the default), initialized to 1,
or it is unitialized.

Syntactically the line in the AIGER format that defines the next state
literal of a latch might now optionally contain an additional literal, which
is either '0', '1', or the literal of the latch itself.  The former are used
for constant initialization and the latter to define an uninitialized latch.

This allows for future more general reset logic without syntactic changes.

\section{Multiple Properties}

A common request from industry was to allow using and checking multiple
properties for the same model.  In practice a model rarely only has one
property and in addition properties like invariants that hold on the model
might help to prove other properties faster.

The extension is rather straigth forward.  Where one property was listed,
i.e. one output in the old format, now multiple properties can be listed.
The major change is in the witness format.  See below for more information
about that.

\section{API}

The library API in 'aiger.h' is extended with functions to support new
features but the already existing functions do not change their meaning.

For the new features the following functions have been added:
{\small
\begin{verbatim}
  void aiger_add_reset (aiger *, unsigned lit, unsigned reset);
  void aiger_add_bad (aiger *, unsigned lit, const char *);
  void aiger_add_constraint (aiger *, unsigned lit, const char *);
  void aiger_add_justice (aiger *, unsigned size, unsigned *, const char *);
  void aiger_add_fairness (aiger *, unsigned lit, const char *);
\end{verbatim}}

There is an additional '\texttt{reset}' field for latches, as well as 
seperate '\texttt{bad}', '\texttt{constraints}', '\texttt{justice}' and
'\texttt{fairness}' sections.  Each section
is a list of '\texttt{aiger\_symbol}' entries:

{\small
\begin{verbatim}
  struct aiger_symbol
  {
    unsigned lit;                 /* unused for justice */
    unsigned next, reset;         /* used only for latches */
    unsigned size, * lits;        /* used only for justice */
    char *name;
  };
\end{verbatim}}

\section{Acknowledgements}

Acknowledgements go to all the supporters of AIGER and the HWMCC.
The new format report version 2.0 will contain a complete list.

\end{document}
